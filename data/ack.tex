%%% Local Variables:
%%% mode: latex
%%% TeX-master: "../main"
%%% End:

\begin{ack}
论文完成之际,回想读博历程。实验进展顺利时的欣喜若狂,理论计算无处入手时的绝望焦躁,熬到深夜周围的一片寂静……一幕幕浮上心头。学习工作中等到了很多人的帮助、支持与鼓励,感激之情无法用语言表达。这里我想用最笨拙的语言表达心中最真挚的感谢。

感谢导师高喆教授给予的教导、支持与帮助。课题研究工作中,高老师给我提出了很多建设性意见,为我明确研究方向提供了指引。并在论文形成撰写过程中付出了很多心血。

感谢蒲以康老师、王文浩老师、谭熠老师、解丽凤老师、彭晓炜老师给予的支持、教导、帮助与鼓励,从他们那里我学到了很多实验方法和技巧。

感谢王龙老师、杨宣宗老师和冯春华老师的关心、爱护与帮助。自本科大学生研究计划开始他们就一直在为我的工作提供帮助和建议,并给我了很多关心和爱护。

感谢张良、曾龙、赵爱慧、刘阳青、姜艳铮、柴忪以及其他师兄弟姐妹的帮助与在实验室的陪伴。

感谢工物系、西南物理研究院、等离子体物理研究所、工程物理研究院给予过我帮助的所有老师和同学。

感谢工作中在仪器制作、安装和调试中给过帮助的所有相关工程技术人员。

感谢我的父母的爱护与支持。尤其感谢张英女士,她的陪伴、支持与照顾是我支持下去的动力。

转眼间何也熙老师离开我们已经 6 年的时间,他为人和蔼,工作敬业,是我学习的榜样。再次对何老师表示深切怀念。

感谢我学习生活中出现过的所有人士。

本课题得到了国家自然科学基金(批准号:10990214、11175103、11261140327、11075092、11005067)和国际热核聚变实验堆(ITER)计划专项课题(批准号:2013GB112001)的资助,在此一并致谢。

感谢 \thuthesis,它的存在让我的论文写作轻松自在了许多,让我的论文格式规整漂亮了许多。
\end{ack}
