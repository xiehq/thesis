\begin{resume}

  \resumeitem{个人简历}

  1984 年 2 月 10 日出生于河北省深州市。

  2002 年 9 月考入中国科学技术大学近代物理系,2006 年 7 月本科毕业并获得理学学士学位。

  2006 年 9 月免试进入清华大学工程物理系攻读核科学与技术博士学位至今。

  \resumeitem{发表的学术论文} % 发表的和录用的合在一起

  \begin{enumerate}[{[}1{]}]
  \item Huiqiao Xie, Zhe Gao, Yi Tan, et al. Electron temperature and density determination in helium plasmas of SUNIST using the optical emission spectrum line-ratio method. The Joint Meeting of 5th IAEA Technical Meeting on Spherical Tori \& 16th International Workshop on Spherical Torus (ISTW2011) \& 2011 US-Japan Workshop on ST Plasma, 2011: Toki.
  \bigskip
  %\vspace{-0.5em}
  \item Xie Huiqiao, Tan Yi, Ke Rui, et al. Analysis of the gas puffng performance for improving the repeatability of Ohmic discharges in the SUNIST spherical tokamak. In press. (已被 Plasma Science and Technology 录用. SCI 源刊.)
  \item 谢会乔, 谭熠, 刘阳青, 等. SUNIST 氦放电等离子体的碰撞辐射模型及其在谱线比法诊断的应用. (已被物理学报录用. SCI 源刊.)
  \end{enumerate}
%  \resumeitem{研究成果} % 有就写,没有就删除
%  \begin{enumerate}[{[}1{]}]
%  \item 任天令, 杨轶, 朱一平, 等. 硅基铁电微声学传感器畴极化区域控制和电极连接的
%    方法: 中国, CN1602118A. (中国专利公开号.)
%  \item Ren T L, Yang Y, Zhu Y P, et al. Piezoelectric micro acoustic sensor
%    based on ferroelectric materials: USA, No.11/215, 102. (美国发明专利申请号.)
%  \end{enumerate}
\end{resume}
